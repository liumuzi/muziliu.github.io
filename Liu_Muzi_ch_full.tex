\documentclass[10pt,a4paper]{article}
\usepackage[UTF8]{ctex}
\usepackage{zhnumber}

\usepackage[a4paper,margin=0.75in,top=0.6in,bottom=0.6in]{geometry}
\usepackage[english]{babel}
\hyphenation{Some-long-word}
\usepackage{resume}
% packages that not in the original template
\usepackage{enumitem}


\begin{document}
\sloppy 


% PERSONAL INFO
\maintitle{刘木子}{}{更新于 \zhtoday}

\nobreakvspace{0.3em}

\noindent\href{mailto:liumuzi.at.link.dot.cuhk.dot.edu.dot.hk}{liumuzi\mbox{}@\mbox{}link.cuhk.edu.hk}\sbull
\textsmaller{+}86.15921643461 \sbull 上海市浦东新区
\\\sbull \href{https://www.linkedin.com/in/muzi-liu-7337b1258}{LinkedIn} \sbull \href{https://github.com/liumuzi}{GitHub} \sbull \href{https://liumuzi.github.io/}{个人网站}

% EXPERIENCE
%\roottitle{Research Experience}
\roottitle{项目与科研经历 Research \& Projects}

\headedsection
  {\emph{独立游戏开发(主程序、策划)}}
  {2024年9月 -- 至今}
  {
  \headedsubsection
    {《我的恋综超失控》(进行中)}
    {}
    {\bodytext{\bull \ \ AI驱动的恋综修罗场:化身导演操控嘉宾,在不可控的即兴互动中制造“抓马”剧情,收割全网流量 \\ 
    \bull \ \ 负责核心程序实现,并参与所有游戏设计决策 \\ 
    \bull \ \ 腾讯游戏创作大赛2025参赛项目,获得AI玩法创作奖
    }}
  \headedsubsection
    {《虫虫家装》(进行中)}
    {}
    {\bodytext{\bull \ \ 温情的装修冒险:在危险的丛林中搜集资源,为水獭家族等小动物装修梦想之家,建立跨种族的羁绊 \\ 
    \bull \ \ 负责核心程序实现,设计装修+搜打撤的核心玩法,并制作原型验证 \\ 
    \bull \ \ 组建六人开发团队,主导团队协作与敏捷开发流程,负责制定里程碑、任务分配与进度管理
    }}
  \headedsubsection
    {\href{https://scheminghare.itch.io/not-again-hero}{《Not Again, Hero》}(Demo)}
    {}
    {\bodytext{\bull \ \ 童话风叙事解密:重组剧情卡片,在无限循环的英雄剧本中,寻找逃离幻想的真相 \\
    \bull \ \ 负责核心程序实现,独立设计核心叙事诡计 \\
    \bull \ \ GMTK2025参赛,叙事性排名\#141/9562(前1\%)
    }}
  \headedsubsection
    {\href{https://www.taptap.cn/app/780672?os=pc}{《报错漫游指南》(Demo)}}
    {}
    {\bodytext{\bull \ \ 平台跳跃解密:程序员半夜加班写代码竟然把自己写进电脑里了!?穿越Bug世界,修复代码逃出程序!
    \bull \ \ 55人下载试玩,评分10/10
    }}
  \headedsubsection
    {\href{https://scheminghare.itch.io/farty-party}{《Farty Party》}(Demo)}
    {}
    {\bodytext{\bull \ \ 无厘头乱斗派对:上演“有屁快放”的抢食大战!利用连环屁、火箭屁轰飞对手,在混乱中争夺冠军 \\
    \bull \ \ 独立设计与开发,一人完成程序、美术与音效制作
    }}
  }

\headedsection
  {\emph{图形算法工程师}}
  {\textsc{米哈游,上海}} {
  \headedsubsection
    {口型动画生成算法}
    {2021年2月 -- 2024年9月}
    {\bodytext{\bull \ \ 维护人声学到口型的Kaldi管线:配音转文字与音素时长对齐,后处理后驱动口型blendshape \\ 
    \bull \ \ 独立负责后处理与算法优化,提升动画效果,并支持多语言口型生成。\\ 
    \bull \ \ 独立负责维护训练与推理服务,保障管线稳定运行;后续端到端模型以该流程为基础\\
    \bull \ \ 在《原神》的制作中投入使用,实现了90\%以上的口型动画全自动生成,无需动画师二次调整。
    }}
  \headedsubsection
    {表情动画生成算法}
    {}
    {\bodytext{\bull \ \ 基于模型完成语音情绪识别;并与动画师协作,按音频区间生成表情动画 \\ 
    \bull \ \ 独立负责情绪识别模块,制定情绪标签体系并优化识别效果 \\ 
    \bull \ \ 参与动画生成工程,实现规则化pose自动排布
    }}
  \headedsubsection
    {动作匹配算法}
    {}
    {\bodytext{\bull \ \ 基于LLM为台词生成动作标签,构建台词-动作自动匹配流程 \\ 
    \bull \ \ 收集数据并调优prompt,提升了动作标签一致性与可用性,减少了策划配置剧情动画的工作量
    }}
  }

\headedsection
  {\emph{实习应用研究员}}
  {\textsc{腾讯光子工作室,深圳}} {
  \headedsubsection
    {《英雄联盟手游》NPC算法}
    {2020年6月 -- 2020年8月}
    {\bodytext{\bull \ \ 主导搭建监督学习训练流程,复现OpenAI Five论文中的核心方法 \\ 
    \bull \ \ 搭建并完善MOBA游戏Demo训练环境,丰富对战场景以支持Agent训练 \\
    \bull \ \ 使用PPO在MOBA环境中训练智能体,并负责模型训练与调优,显著提升了智能体的表现
    }}
  }

\headedsection
  {\emph{助理研究员}}
  {\textsc{香港中文大学,香港}} {
  \headedsubsection
    {使用强化学习完成的语汇合作游戏(Lexicon Coordination Game)}
    {2018年1月 -- 2019年7月}
    {\bodytext{\bull \ \ 在语汇合作游戏中,提出新的强化学习方法,发现了网络中局部语汇共识对整体语汇共识的阻碍 \\ 
    \bull \ \ 独立写作\href{https://dl.acm.org/doi/abs/10.1007/978-3-030-61401-0_64}{论文},发表于ICAISC 2020会议
    }}
  
  \headedsubsection
    {\href{http://www.eim.hk/}{运动是良药 Exercise Is Medicine (EIM):慢性病患者的运动监督程序}}
    {2019年5月 -- 2019年9月}
    {\bodytext{\bull \ \ 训练LSTM神经网络,检测运动手环佩戴位置,提升运动量估算的准确度
    }}
  
  \headedsubsection
    {\href{https://funtomove-jc.hk/zh/}{家校童喜动 FunToMove @ JC: 学童运动监督程序}}
    {}
    {\bodytext{\bull \ \ 使用 TypeORM 和 MSSQL 数据库,开发后端 RESTful API 
    }}
  }


\vspace{-0.9em}

% EDUCATION
\roottitle{教育背景 Education}

\headedsection
  {\href{http://www.cuhk.edu.hk/english/index.html}{香港中文大学}}
  {\textsc{中国香港}} {
  \headedsubsection
    {\emph{学士} 计算机科学 (人工智能方向)}
    {2015 -- 2020}
    {\bodytext{\bull \ \ 专业绩点: 3.70 / 4.00 \sbull 工程学院院长名单 (学院前 10\%)
    }}
  }

\headedsection
  {\href{https://www.cmc.edu/}{克莱蒙特麦肯纳夫人学院}}
  {\textsc{美国加州}} {%
  \headedsubsection
    {\emph{交换学生}}
    {2017 -- 2018}
    {\bodytext{\bull \ \ Introduction to Psychology (A-) \bull \ \ Neuropsychology (A-)}}
}
\vspace{-0.9em}

% SKILLS
\roottitle{语言与技能 Skills}

\headedsection{专业技能:}{}{
\emph{编程语言:} C\# (Unity), Python, C/C++, Java\\
\emph{算法与应用:} 强化学习、语音/情绪识别、口型/表情/动作生成\\
\emph{工程与数据:} 模型训练与调优、数据收集与标注、服务部署与维护、数据库 (SQL/NoSQL)
}

\headedsection{语言:}{}{
普通话(母语),粤语(基础),英语(TOEFL 106/120)及日语(基础)}
\vspace{-0.9em}


% HONORS
%\roottitle{获奖情况 Honors \& Awards}
%学期交换奖学金 2016/17 \hfill 香港中文大学,伍宜孙书院\\
%康本国际交流奖学金 \hfill 香港中文大学,学术链接办公室(OAL)\\
%新生入学奖学金 \hfill 香港中文大学,工程学院\\
%\vspace{-1.3em}

\end{document}